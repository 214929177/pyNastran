\section{OP2}
The OP2 only supports PARAM,POST,-1.  Be careful on which output argument you use.

\subsection{BDF Tables}
  Chances are you're not too interested in the BDF, but if you ever need it,
  you can extract it from the OP2.  Note, this feature is not fully implemented.
 \begin{tabular}{ll}
    GEOM1            & Nodes, coordinate Systems         \\
    GEOM2            & Elements                          \\
    GEOM3            & Loads                             \\
    GEOM4            & Constraints                       \\
    EPT/EPTS         & Properties                        \\
    MPT/MPTS         & Materials                         \\
    PCOMP/PCOMPTS    & ???                   \\
    DYNAMIC/DYNAMICS & Dynamic Cards         \\
    DESTAB           & Design Variable Table \\
    CASECC           & Case Control          \\
    DIT              & Tables (e.g. TABLEM1) \\
\end {tabular}

\subsection{Aero Tables}
 This is useful for Aero calculations, but it's not supported.
 \begin{tabular}{ll}
    MONITOR    & Monitor Point ???             \\
    AEMONPT    & Aeroelastic Monitor Point ??? \\
 \end {tabular}

\subsection{Results}
 Standard result tables are organized into a few categories
 \begin{tabular}{ll}
    U   & Displacment/Temperature    \\
    Q   & SPC/MPC Constraint Forces  \\
    P   & Applied Loads              \\
    V   & Velocity                   \\
    A   & Acceleration               \\
 \end {tabular}

   These form the OUG, OQG, OUP, OUV, OUA tables.  
   
  \subsubsection{Sort Codes}
   Tables may have modifiers that indicate the format of their transient output.
      - SORT1 - loop over $\delta t$, loop over node ID
      - SORT2 - loop over node ID, loop over $\delta t$
   Note that $\delta t$ may be swapped based on the solution type.  For example, 
   $\delta t$ may instead be frequency, eigenvalue, load step, etc.  
   This concept is important for understanding the format of OP2/F06 results.
   
   By adding the SORT1 and SORT2 terms, we create the following OQG tables.
   Luckily, all these tables have the same basic format.
   There are similar SORT2 tables.
   
   Note the OUG, OUV, and OUA tables are all part of oug.py because
   they're so similar.

   \begin{tabular}{ll}
    Table Name & Description             \\
    OQG1   & Constraint Forces           \\
    OQGV1  & SPC ??? Constraint Forces   \\
    OQMG1  & MPC Constraint Forces       \\
   \end {tabular}

   \subsubsection{Format Codes}
    The "format" of the data can also be different.  The formats are:
   \begin{tabular}{ll}
    Format Code & Description                  \\
    0           & Real Results                 \\
    1           & Complex Results              \\
    2           & Random Real/Complex Results  \\
   \end {tabular}
   You can output "Real Displacment", "Complex Displacement", or when performing
   a frequency/PSD/RMS/etc. analysis, "Displacment vs Frequency (Magnitude + Phase)",
   or "Displacment vs Frequency (Real + Imaginary)".
   

 \begin{tabular}{ll}
    HISADD   & Solution 200 Convergence History??? \\
    OGPWG    & Grid Point Weight      \\
    OGS      & Grid Point Stresses    \\
    OGP      & Grid Point Forces      \\

    LAMA     & Eigenvector            \\
    CLAMA    & Complex Eigenvector    \\

    OUG      & Displacement, Temperature, Velocity, Acceleration   \\
    OQG      & SPC/MPC Forces                                      \\
    OES      & Stress/Strain          \\
    OSTR     & Stress/Strain          \\
    OEF      & Applied Forces         \\
    ONR      & Strain Energy Density  \\
\end {tabular}

%DIT,  # tables
%BGPDT,EQEXIN/EQEXINS,PVT0,EDOM,
%R1TABRG
%VIEWTB,ERROR,
%OMM2
%EDOM
%STDISP,SDF

