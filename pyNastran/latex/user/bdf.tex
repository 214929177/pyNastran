
\title{pyNastran Theory Document \\
\small http://code.google.com/p/pynastran/ }
\author{Steven P. Doyle\\
{\small mesheb82@gmail.com}
}

\maketitle

\begin{abstract}
This document is intended to be a reference guide for users.
\end{abstract}

Copyright \copyright\ Steve P. Doyle 2011-2012
\newpage

\tableofcontents
\newpage

\section{BDF}

 \subsection{Nodes}
  \subsubsection{GRID}
     In NASTRAN, nodes can be in the global or a local coordinate
     frame (rectangular, cylindrical, spherical).  Each node may reference
     a different coordinate frame $cp$ for the reference coordinate frame
     (what the value is in the $x_1$, $x_2$, and $x_3$ fields) and for the
     analysis coordinate frame $aid$.
     
     Node Specific Variable Names:
      - xyz (Raw location <$x_1$, $x_2$, $x_3$> in the BDF)
      - cp (reference coordinate system)
      - cd (analysis  coordinate system)
      - Position (location in an arbitrary coordinate system)
      - UpdatePosition (change the value of xyz and cp)
      - resolveGrids (update all the positions of the grids to a common coordinate system in an more efficient way)

     Using the node object:
      - bdf = BDF()
      - bdf.readBDF(bdfName)
      - node = bdf.Node(nid)
      - node.Position()          % gets the position of the node in the global frame
      - node.Position(cid=0)    % same
      - node.Position(cid=1)    % gets the position of the node in a local frame
      - node.UpdatePosition(bdf,array([1.,2.,3.]),cid=3) % change the location of the node
      - bdf.resolveGrids(cid=0) % change the xyz of all nodes to the same coordinate system as cid
      - bdf2 = BDF()
      - bdf2.readBDF(bdfNameAlt)
      - bdf.unresolveGrids(bdf2) % change the coordinte system back to the coordinate system in bdf2
      
        
  \subsubsection{CORDx}
     A coordinate system may be defined by 3 points in any non-circular
     coordinate system or by 3 vectors.
     
     Once cross-referenced a node may use it's node.Position(cid) method to
     determine the position in any frame defined by a coordinate system.
  
  \subsubsection{asdf}
     asfd

