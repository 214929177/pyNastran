
\title{pyNastran Theory Document \\
\small https://github.com/SteveDoyle2/pyNastran }
\author{Steven P. Doyle\\
{\small mesheb82@gmail.com}
}

\maketitle

\begin{abstract}
This document is intended to be a reference guide for users.
\end{abstract}

Copyright \copyright\ Steve P. Doyle 2011-2015
\newpage

\tableofcontents
\newpage

\section{BDF}

 \subsection{Nodes}
  \subsubsection{GRID}
     In NASTRAN, nodes can be in the global or a local coordinate
     frame (rectangular, cylindrical, spherical).  Each node may reference
     a different coordinate frame $cp$ for the reference coordinate frame
     (what the value is in the $x_1$, $x_2$, and $x_3$ fields) and for the
     analysis coordinate frame $aid$.
     
     Node Specific Variable Names:
      - xyz (Raw location <$x_1$, $x_2$, $x_3$> in the BDF)
      - cp (reference coordinate system)
      - cd (analysis  coordinate system)
      - seid (superelement id)

     Node Methods
      - Position -> get_position (location in the global coordinate system)
      - get_position_wrt  (location in an arbitrary coordinate system)
      - resolveGrids -> resolve_grids (update all the positions of the grids to a common coordinate system in an more efficient way)

     Using the node object:
      - bdf = BDF()
      - bdf.read_bdf(bdfName)
      - node = bdf.Node(nid)
      - node.get_position()            % gets the position of the node in the global frame
      - node.get_position_wrt(cid=0)   % same
      - node.get_position_wrt(cid=1)   % gets the position of the node in a local frame
      - bdf.resolve_grids(cid=0) % change the xyz of all nodes to the same coordinate system as cid
      - bdf2 = BDF()
      - bdf2.read_bdf(bdfNameAlt)

     Do these work?
      - node.update_position(bdf, array([1.,2.,3.]), cid=3) % change the location of the node
      - bdf.unresolve_grids(bdf2) % change the coordinte system back to the coordinate system in bdf2
      
        
  \subsubsection{CORDx}
     A coordinate system may be defined by 3 points in any non-circular
     coordinate system or by 3 vectors.
     
     Once cross-referenced a node may use it's node.Position(cid) method to
     determine the position in any frame defined by a coordinate system.
  
  \subsubsection{asdf}
     asfd

